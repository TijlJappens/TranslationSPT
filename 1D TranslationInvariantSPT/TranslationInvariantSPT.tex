\documentclass[12pt,a4paper,twoside]{article}
\usepackage{graphicx,xcolor,textpos}
\usepackage{helvet}
\usepackage[english]{babel}
\usepackage{amsmath}
\usepackage{amsthm}
\usepackage{bbm}
\usepackage{amssymb}
\usepackage{hyperref}
\usepackage{relsize}
\usepackage[margin=0.7in]{geometry}
\usepackage{physics}
\usepackage{enumitem}
\usepackage{mathtools}
\usepackage{changepage}
\usepackage{caption}
\usepackage{subcaption}
\usepackage{verbatim}
\usepackage{url}
\usepackage{tikz}
\usetikzlibrary{calc}

\renewcommand{\d}{\text{d}}
\renewcommand{\O}{\mathcal{O}}
\newcommand{\e}{\mathlarger{e}}
\newcommand{\defeq}{\vcentcolon=}
\let\originalleft\left
\let\originalright\right
\renewcommand{\left}{\mathopen{}\mathclose\bgroup\originalleft}
\renewcommand{\right}{\aftergroup\egroup\originalright}
\title{1D SPT classification with translation symmetry}
\author{Tijl Jappens}
\date{\today}

\newcommand{\UU}{\mathcal U}
\newcommand{\KK}{\mathcal K}
\newcommand{\BB}{\mathcal B}
\newcommand{\PP}{\mathcal P}
\newcommand{\HH}{\mathcal H}
\newcommand{\ZZ}{\mathbb Z}
\newcommand{\CC}{\mathbb C}
\newcommand{\TT}{\mathbb T}
\renewcommand{\AA}{\mathcal A}
\newcommand{\LL}{\mathcal L}
\newcommand{\RR}{\mathbb R}
\newcommand{\NN}{\mathbb{N}}
\newcommand{\one}{\mathbbm{1}}

\newcommand{\Ad}{\textrm{Ad}}

\newcommand{\qe}{\underset{\text{q.e.}}{\sim}}

\newcommand{\Mod}[1]{\mathrm{mod} #1}

\theoremstyle{definition}
\newtheorem{theorem}{Theorem}[section]
\newtheorem{definition}[theorem]{Definition}
\newtheorem{lemma}[theorem]{Lemma}
\newtheorem{remark}[theorem]{Remark}

\numberwithin{equation}{section}
\begin{document}
\section{example}
One particularly simple example is that of matrix product states where the matrices at each site are the same. In general, if the matrices that make up the matrix product state are $\Lambda_i$ they transform as
\begin{equation}
	\sum_{i=1}^d U(g)_{ij}\Lambda_j=\alpha(g) R^{-1}(g)\Lambda_i R(g)
\end{equation}
where $d$ is the dimension of the on site Hilbert space, $R$ is a lift of the projective representation that eventually gives the $H^2$ index and $\alpha$ is the $U(1)$ representation that gives the translation index.
\section{Setup}
Take $\AA$ to be the 1D quasi local $C^*$ algebra generated by taking the infinite tensor product of the on site algebra $\BB(\CC^d)$. Take $G$ a finite group. Take $U\in\hom(G,\UU(\CC^d))$ to be the on site group action which we take to be uniform in space and take $\beta\in\hom(G,\text{Aut}(\AA))$ to be the corresponding group action on the algebra. For any subset $I\in\ZZ$ we will write $\beta_I$ to mean the restriction of $\beta$ to $\AA_I$ and for any finite interval $I\subset\ZZ$ we will write $U_I(g)=\otimes_{i\in I}U(g)\in\UU(\AA_I)$. In our convention we take $\beta_I(g)=\Ad(U_I(g))$. Take $\tau:\ZZ\rightarrow\text{Aut}(\AA)$ to be the action of the translation on $\AA$ where $\tau(+1)$ will mean a translation to the right. Clearly by construction we have that $\tau(z)\circ\beta(g)=\beta(g)\circ\tau(z)$. In this paper we choose the adjoint to be such that $\Ad(u)(v)=uvu^\dagger$.
\begin{definition}
	Interaction...
\end{definition}
\begin{definition}
	f-norm...
\end{definition}
\begin{definition}
	short range entangled and invertible...\\
	Do we ask that the product state related to a translation invariant invertible state also be translation invariant? I guess there is a similar question about on site group actions.
\end{definition}
\section{Definition of the index}
The index we propose does not require our state to be invertible but only requires it to have a strictly weaker property namely the split property. The things we will use from the split property are written in the following lemma
\begin{lemma}
	Take $\omega\in\PP(\AA)$ such that
	\begin{enumerate}
		\item $\omega$ satisfies the split property.
		\item $\omega\circ\beta(g)=\omega\quad\forall g\in G$.
		\item $\omega\circ\tau(z)=\omega\quad\forall z\in\ZZ$.
	\end{enumerate}
	The following now holds:
	\begin{enumerate}
		\item There is a GNS triple of the form
		\begin{equation}
			(\HH=\HH_L\otimes\HH_R,\pi=\pi_L\otimes\pi_R,\Omega_\omega)
		\end{equation}
		where $\pi_{L/R}$ is a function of $\AA_{L/R}$ only. We will call such a GNS triple a split GNS triple.
		\item There exist maps
		\begin{equation}
			u_{L/R}:G\rightarrow\UU(\HH_{L/R})
		\end{equation}
		such that:
		\begin{itemize}
			\item $u_L\otimes u_R\in\hom(G,\UU(\HH))$.
			\item $\pi_{L/R}\circ \beta_{L/R}(g)=\Ad(u_{L/R}(g))\circ\pi_{L/R}$.
			\item $u_L(g)\otimes u_R(g)\Omega_\omega=\Omega_\omega\quad\forall g\in G$.
		\end{itemize}
		Given a split GNS triple, these maps are unique up to a $g$-dependent phase ($u_L\otimes u_R$ is then truly unique).
		\item There is a unique (again after fixing a split GNS triple) $v\in\UU(\HH)$ such that
		\begin{align}
			\pi\circ\tau(z)&=\Ad(v^z)\circ\pi&\text{and}&&v^z\Omega_\omega&=\Omega_\omega.
		\end{align}
	\end{enumerate}
\end{lemma}
\begin{proof}
	This all feels rather standard.
\end{proof}
Sometimes, depending on the context we will write $u_{L/R}$ to mean $u_L\otimes\one$ and $\one\otimes u_R$ respectively. The following lemma will also be the definition of our index:
\begin{lemma}
	There exists an $\alpha\in\hom(G,U(1))$ such that
	\begin{equation}
		v u_R(g)v^\dagger =\alpha(g) u_R(g)\pi(U_0(g))^\dagger
	\end{equation}
	$\forall g\in G$. This $\alpha$ is independent of the choice of split GNS triple.
\end{lemma}
\begin{proof}
	We need to prove that both operators are equal in the entire GNS Hilbert space. This is equivalent to showing that for each $a\in\AA$ we have that
	\begin{equation}\label{eq:equalityIndexInGNSSpace}
		v u_R(g)v^\dagger \pi(a)\Omega_\omega =\alpha(g) u_R(g)\pi(U_0(g))^\dagger\pi(a)\Omega_\omega.
	\end{equation}
	We will split the proof in three parts. First we show that
	\begin{equation}\label{eq:equalityIndexAsAdjoint}
		\Ad(v u_R(g)v^\dagger)\circ\pi=\pi\circ\beta_{[1,\infty[}(g)
	\end{equation}
	then using this we show that equality \eqref{eq:equalityIndexInGNSSpace} is true for $a=\one$, then finally using both previous items we show equation \eqref{eq:equalityIndexInGNSSpace} for general $a\in\AA$. For the first part showing equation \eqref{eq:equalityIndexAsAdjoint} we have that
	\begin{align}
		\Ad(v u_R(g)v^\dagger)\circ\pi&=\Ad(v)\circ\Ad(u_R(g))\circ\Ad(v^\dagger)\circ\pi\\
		&=\pi\circ\tau(+1)\circ\beta_{[0,\infty[}(g)\circ\tau(-1)\\
		&=\pi\circ\beta_{[1,\infty[}(g)\\
		&=\pi\circ\beta_0(g)^{-1}\circ\beta_{[0,\infty[}(g)\\
		&=\Ad(u_R(g))\circ\Ad(\pi(U_0(g))^\dagger)\circ\pi\\
		&=\Ad\left(u_R(g)\pi(U_0(g))^\dagger\right)\circ\pi.
	\end{align}
	For the second part we will start by showing that the following two GNS triples
	\begin{align}\label{eq:TwoGNSTriples}
		\left(\HH,\pi,v u_R(g)v^\dagger \Omega_\omega\right)&&\left(\HH,\pi,u_R(g)\pi(U_0(g))^\dagger \Omega_\omega\right)&
	\end{align}
	are unitary equivalent. Notice that by the irreducibility of the GNS representation that this implies that the two vectors are equal up to a phase. We have that
	 \begin{align}
	 	&(\HH,\pi,v u_R(g)v^\dagger \Omega_\omega)\\
	 	&\sim (\HH,\Ad(v u_R(g)^\dagger v^\dagger)\circ\pi,\Omega_\omega)=(\HH,\Ad(\pi(U_0(g))u_R(g)^\dagger)\circ\pi,\Omega_\omega)\\
	 	&\sim (\HH,\pi,u_R(g)\pi(U_0(g))^\dagger \Omega_\omega)
	 \end{align}
 	where we've used equation \eqref{eq:equalityIndexAsAdjoint}. At this point we have established that there exists an $\alpha:G\rightarrow U(1)$ (not yet that it is a homomorphism) such that
 	\begin{equation}
 		v u_R(g) v^\dagger \pi(a) \Omega_\omega =\alpha(g) u_R(g) \pi(U_0(g))\pi(a) \Omega_\omega
 	\end{equation}
 	for $a=\one$ now we will extend this fact to all $a\in\AA$. We have
 	\begin{align}
 		v u_R(g) v^\dagger \pi(a) \Omega_\omega&=\Ad(v u_R(g) v^\dagger)\circ \pi(a) v u_R(g) v^\dagger \Omega_\omega\\
 		&=\alpha(g) \Ad(u_R(g) \pi(U_0(g)))\circ \pi(a) u_R(g) \pi(U_0(g)) \Omega_\omega.
 	\end{align}
 	This should indeed prove that on the full GNS Hilbert space
 	\begin{equation}
 		v u_R(g)v^\dagger =\alpha(g) u_R(g)\pi(U_0(g))^\dagger.
 	\end{equation}
 	Because this equation is true it now follows as well that $\alpha$ has to be a homomorphism. The independence of choice of split GNS triple seems rather straightforward.
\end{proof}
%\begin{lemma}
%	The map
%	\begin{equation}
%		h:G\rightarrow U(1):g\mapsto \bra{\Omega_\omega}\ket{v^\dagger u_R(g)^\dagger \pi(U_0(g)) v u_R(g)\Omega_\omega}
%	\end{equation}
%	is a homeomorphism.
%\end{lemma}
%\begin{proof}
%	content...
%\end{proof}
\bibliography{FSPT}{}
\bibliographystyle{plain}
\end{document}