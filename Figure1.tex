\documentclass[preview=false]{standalone}
\begin{document}

\begin{figure}
	\centering
	\def\s{0.5}
	\resizebox{0.45\textwidth}{!}{%
		\begin{tikzpicture}
			\fill[fill=blue!30!white] (-4*\s,-4*\s) rectangle (4*\s,4*\s);
			\draw[draw=black,line width=0.3mm] (0,4*\s) -- (0,-4*\s);
			
			\node at (-2*\s,0) {$\alpha_L$};
			\node at (2*\s,0) {$\alpha_R$};
		\end{tikzpicture}}
	$\qquad$
	\def\s{0.4}
	\resizebox{0.45\textwidth}{!}{%
		\begin{tikzpicture}
			\draw[draw=white,line width=0mm] (1,0) coordinate (a) -- (0,0) coordinate (b) -- (1,1) coordinate (c);
			
			\fill[fill=green!30!white] (0,0) -- (-5*\s,-5*\s) -- (-5*\s,5*\s);
			\fill[fill=green!30!white] (0,0) -- (5*\s,-5*\s) -- (5*\s,5*\s);
			\fill[fill=red!30!white] (0,1*\s) -- (4*\s,5*\s) -- (-4*\s,5*\s);
			\fill[fill=red!30!white] (0,-1*\s) -- (4*\s,-5*\s) -- (-4*\s,-5*\s);
			
			\draw[draw=black,line width=0.3mm]
			plot[smooth,samples=2,domain=0:4*\s] (\x,\x+1*\s);
			\draw[draw=black,line width=0.3mm]
			plot[smooth,samples=2,domain=-4*\s:0] (\x,-\x+1*\s);
			\draw[draw=black,line width=0.3mm]
			plot[smooth,samples=2,domain=0:4*\s] (\x,-\x-1*\s);
			\draw[draw=black,line width=0.3mm]
			plot[smooth,samples=2,domain=-4*\s:0] (\x,\x-1*\s);
			\draw[draw=black,line width=0.3mm]
			plot[smooth,samples=2,domain=-5*\s:5*\s] (\x,\x);
			\draw[draw=black,line width=0.3mm]
			plot[smooth,samples=2,domain=-5*\s:5*\s] (\x,-\x);
			
			\draw[draw=black,line width=0.3mm] (0,0) -- (2*\s,0);
			\pic [draw, ->, "$\theta$", angle eccentricity=1.5, angle radius=\s*1.5cm,line width=0.3mm] {angle=a--b--c};
			
			\node at (-3*\s,0) {$\eta^L_g$};
			\node at (3.5*\s,0) {$\eta^R_g$};
			\node at (0,3*\s) {$\Theta$};
	\end{tikzpicture}}
	\caption{These figures indicate the support area of the different automorphisms. The angle $\theta$ needs to be smaller then or equal to what was indicated here so that the $\Theta$ and the $\eta_g$ (possibly after widening by one site) commute.}
	\label{fig:SetupWithQAutomorphism}
\end{figure}
\end{document}