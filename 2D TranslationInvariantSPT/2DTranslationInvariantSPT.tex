\documentclass[12pt,a4paper,twoside]{article}
\usepackage{graphicx,xcolor,textpos}
\usepackage{helvet}
\usepackage[english]{babel}
\usepackage{amsmath}
\usepackage{amsthm}
\usepackage{bbm}
\usepackage{amssymb}
\usepackage{hyperref}
\usepackage{relsize}
\usepackage[margin=0.7in]{geometry}
\usepackage{physics}
\usepackage{enumitem}
\usepackage{mathtools}
\usepackage{changepage}
\usepackage{caption}
\usepackage{subcaption}
\usepackage{verbatim}
\usepackage{url}
\usepackage{tikz}
\usetikzlibrary{calc}

\renewcommand{\d}{\text{d}}
\renewcommand{\O}{\mathcal{O}}
\newcommand{\e}{\mathlarger{e}}
\newcommand{\defeq}{\vcentcolon=}
\let\originalleft\left
\let\originalright\right
\renewcommand{\left}{\mathopen{}\mathclose\bgroup\originalleft}
\renewcommand{\right}{\aftergroup\egroup\originalright}
\title{1D SPT classification with translation symmetry}
\author{Tijl Jappens}
\date{\today}

\newcommand{\UU}{\mathcal U}
\newcommand{\KK}{\mathcal K}
\newcommand{\BB}{\mathcal B}
\newcommand{\PP}{\mathcal P}
\newcommand{\HH}{\mathcal H}
\newcommand{\ZZ}{\mathbb Z}
\newcommand{\CC}{\mathbb C}
\newcommand{\TT}{\mathbb T}
\renewcommand{\AA}{\mathcal A}
\newcommand{\LL}{\mathcal L}
\newcommand{\RR}{\mathbb R}
\newcommand{\NN}{\mathbb{N}}
\newcommand{\one}{\mathbbm{1}}

\newcommand{\Ad}[1]{\textrm{Ad}\left(#1\right)}
\newcommand{\Aut}[1]{\textrm{Aut}\left(#1\right)}

\newcommand{\qe}{\underset{\text{q.e.}}{\sim}}

\newcommand{\Mod}[1]{\mathrm{mod} #1}

\theoremstyle{definition}
\newtheorem{theorem}{Theorem}[section]
\newtheorem{definition}[theorem]{Definition}
\newtheorem{lemma}[theorem]{Lemma}
\newtheorem{remark}[theorem]{Remark}

\numberwithin{equation}{section}
\begin{document}
Take $\omega\in\PP(\AA)$ to be such that
\begin{enumerate}
	\item  there exists an automorphism $\alpha\in\Aut{\AA}$ such that
	\begin{align}
		\omega&=\omega_0\circ\alpha&\alpha&=\Ad{V_1}\circ\alpha_L\otimes\alpha_R\circ\Theta
	\end{align}
	where $\Theta\in\Aut{\AA_{(C_\theta\cup C_\theta')^c}}$\footnote{By this I mean it is supported in the complement of the union of the horizontally oriented cone through the origin with angle $\theta$ and the cone that is shifted one site upwards (through site $(0,1)$)}, $\alpha_L\in\Aut{\AA_L}$, $\alpha_R\in\Aut{\AA_R},V_1\in\UU(\AA)$ and $\omega_0\in\PP(\AA)$ is a product state.
	\item there exists a map
	\begin{equation}
		\tilde\beta:G\rightarrow\Aut{\AA}:g\mapsto\tilde\beta_g
	\end{equation}
	satisfying
	\begin{align}
		\omega\circ\tilde\beta_g&=\omega&\tilde\beta_g=\Ad{V_2}\circ\eta^L_g\otimes\eta^R_g\circ\beta^U_g
	\end{align}
	where $V_2\in\UU(\AA),\eta^L_g\in\Aut{\AA_L\cap\AA_{C_\theta}}$ and $\eta^R_g\in\Aut{\AA_R\cap\AA_{C_\theta}}$.
\end{enumerate}
Take $(\HH_0=\HH_L\otimes\HH_R,\pi_0=\pi_L\otimes\pi_R,\Omega_L\otimes\Omega_R)$a GNS triple of $\omega_0$. Now use lemma 2.1 from Yoshiko Ogata \cite{ogata2021h3gmathbb} to define the objects $W_g$ and $u_\sigma(g,h)$ they will be the starting point for our definition. We now define the translation action on the GNS space:
\begin{lemma}
	There exists a unique $v\in\UU(\HH_0)$ such that
	\begin{enumerate}
		\item $\Ad{v}\circ\pi_0=\pi_0\circ\alpha_0\circ\Theta\circ\tau\circ\Theta^{-1}\circ\alpha_0^{-1}.$
		\item $\pi_0(V_1)v\pi_0(V_1^\dagger)\Omega_0=\Omega_0.$
	\end{enumerate}
\end{lemma}
We now have to define one more object
\begin{lemma}
	There exist $K_g^L\in\UU(\HH_L)$ and $K_g^R\in\UU(\HH_R)$ such that
	\begin{equation}
		v^\dagger W_g v W_g^\dagger=K_g^L\otimes K_g^R=K_g
	\end{equation}
	and they are unique (up to a phase). They satisfy the identity
	\begin{equation}
		\Ad{K^R_g}\circ\pi_0=\pi_0\circ \alpha_0\circ \tau^{-1}\circ \eta_g^R\circ\beta_g^{RU}\circ\tau\circ(\beta_g^{RU})^{-1}\circ(\eta_g^{R})^{-1}\circ\alpha_0^{-1}.\quad\footnote{Observe in particular that $\Theta$ has cancelled out}
	\end{equation}
\end{lemma}
\begin{proof}
	content...
\end{proof}
\begin{lemma}
	Take $\gamma_{L/R}\in\Aut{(\AA_{C_\theta}\cup\AA_{C_\theta'})^c\cap\AA_{L/R}}$ and take $\Gamma_{L/R}$ to be such that
	\begin{equation}
		\pi_{L/R}\circ\alpha_{L/R}\circ\gamma_{L/R}\circ\alpha_{L/R}^{-1}= \Ad{\Gamma_{L/R}}\circ\pi_{L/R}
	\end{equation}
	then
	\begin{equation}
		\Ad{W_g}(\Gamma_{L}\otimes\Gamma_{R})=\Ad{W_g}(\Gamma_L)\otimes\Ad{W_g}(\Gamma_{R})
	\end{equation}
	and moreover there exist $\tilde\gamma_{L/R}\in\Aut{(\AA_{C_\theta}\cup\AA_{C_\theta'})^c\cap\AA_{L/R}}$ such that
	
\end{lemma}
\begin{lemma}
	The following identity holds:
	\begin{equation}
		\Ad{K_g^R\Ad{W_g}\left(K_h^R\Ad{W_h}\left((K_{gh}^R)^\dagger\right)\right)u_
		R(g,h)}\circ\pi_0=\Ad{v^\dagger u_R(g,h)v}\circ\pi_0.
	\end{equation}
\end{lemma}
\begin{proof}
	We first prove the result for the full tensor product:
	\begin{align}
		&\Ad{v^\dagger u_L(g,h)\otimes u_R(g,h) v}\circ\pi_0\\
		&=\Ad{v^\dagger W_g W_h W_{gh}^{-1}v}\circ\pi_0\\
		&=\Ad{K_g W_gK_h W_hK_{gh}^{-1}W_gW_hW_{gh}^{-1}}\circ\pi_0\\
		&=\Ad{K_g\Ad{W_g}\left(K_h\Ad{W_h}\left(K_{gh}^\dagger\right)\right)u_L(g,h)\otimes u_
			R(g,h)}\circ\pi_0.
	\end{align}
	
\end{proof}
\bibliography{TSPT}
\bibliographystyle{plain}
\end{document}