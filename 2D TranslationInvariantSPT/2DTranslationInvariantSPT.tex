\documentclass[12pt,a4paper,twoside]{article}
\usepackage{graphicx,xcolor,textpos}
\usepackage{helvet}
\usepackage[english]{babel}
\usepackage[normalem]{ulem}
\usepackage{amsmath}
\usepackage{amsthm}
\usepackage{bbm}
\usepackage{amssymb}
\usepackage{hyperref}
\usepackage{relsize}
\usepackage[margin=0.7in]{geometry}
\usepackage{physics}
\usepackage{enumitem}
\usepackage{mathtools}
\usepackage{changepage}
\usepackage{caption}
\usepackage{subcaption}
\usepackage{verbatim}
\usepackage{url}
\usepackage{tikz}
\usetikzlibrary{calc}


\newcommand{\stkout}[1]{\ifmmode\text{\sout{\ensuremath{#1}}}\else\sout{#1}\fi}
\renewcommand{\d}{\text{d}}
\renewcommand{\O}{\mathcal{O}}
\newcommand{\e}{\mathlarger{e}}
\newcommand{\defeq}{\vcentcolon=}
\let\originalleft\left
\let\originalright\right
\renewcommand{\left}{\mathopen{}\mathclose\bgroup\originalleft}
\renewcommand{\right}{\aftergroup\egroup\originalright}
\title{1D SPT classification with translation symmetry}
\author{Tijl Jappens}
\date{\today}

\newcommand{\UU}{\mathcal U}
\newcommand{\KK}{\mathcal K}
\newcommand{\BB}{\mathcal B}
\newcommand{\PP}{\mathcal P}
\newcommand{\HH}{\mathcal H}
\newcommand{\ZZ}{\mathbb Z}
\newcommand{\CC}{\mathbb C}
\newcommand{\TT}{\mathbb T}
\renewcommand{\AA}{\mathcal A}
\newcommand{\LL}{\mathcal L}
\newcommand{\RR}{\mathbb R}
\newcommand{\NN}{\mathbb{N}}
\newcommand{\one}{\mathbbm{1}}

\newcommand{\Ad}[1]{\textrm{Ad}\left(#1\right)}
\newcommand{\Aut}[1]{\textrm{Aut}\left(#1\right)}
\newcommand{\QAut}[1]{\textrm{QAut}\left(#1\right)}

\newcommand{\qe}{\underset{\text{q.e.}}{\sim}}

\newcommand{\Mod}[1]{\mathrm{mod} #1}

\theoremstyle{definition}
\newtheorem{theorem}{Theorem}[section]
\newtheorem{definition}[theorem]{Definition}
\newtheorem{lemma}[theorem]{Lemma}
\newtheorem{remark}[theorem]{Remark}

\numberwithin{equation}{section}
\begin{document}
\section{Setup and definitions}
The following class of automorphisms will be used to define the class of states considered in this paper.
\begin{definition}
	Take $\alpha\in\Aut{\AA}$. We say that $\alpha\in\QAut{\AA}$ if and only if $\forall\theta\in]0,\pi/2[$ there exists an $\alpha_L\in\Aut{\AA_L},\alpha_R\in\Aut{\AA_R},V_1\in\UU(\AA)$ and a $\Theta\in\Aut{(\AA_{C_\theta} \cup \tau(\AA_{C_\theta}))^c}$ such that
	\begin{equation}
	\alpha=\Ad{V_1}\circ\alpha_L\otimes\alpha_R\circ\Theta.
	\end{equation}
\end{definition}
We will use this class of automorphisms to define the states for which we can define the index. Take $\omega\in\PP(\AA)$ to be such that
\begin{enumerate}
	\item  there exists an automorphism $\alpha\in\QAut{\AA}$ satisfying
	\begin{align}
		\omega=\omega_0\circ\alpha.
	\end{align}
	\item $\forall \theta\in]0,\pi/2[$ there exists a map
	\begin{equation}
		\tilde\beta:G\rightarrow\Aut{\AA}:g\mapsto\tilde\beta_g
	\end{equation}
	satisfying
	\begin{align}
		\omega\circ\tilde\beta_g&=\omega&\tilde\beta_g=\Ad{V_2}\circ\eta^L_g\otimes\eta^R_g\circ\beta^U_g
	\end{align}
	where $V_{g,2}\in\UU(\AA),\eta^L_g\in\Aut{\AA_L\cap\AA_{C_\theta}}$ and $\eta^R_g\in\Aut{\AA_R\cap\AA_{C_\theta}}$.
\end{enumerate}
Take $(\HH_0=\HH_L\otimes\HH_R,\pi_0=\pi_L\otimes\pi_R,\Omega_0)$ a GNS triple of $\omega_0$. Following Yoshiko Ogata \cite{ogata2021h3gmathbb} we now define objects using the following lemma:
\begin{lemma}\label{lem:Definition_W_And_u}
	There unitaries $W_g\in\UU(\HH_0)$ and $u_{\sigma}(g,h)\in\UU(\HH_{\sigma})$ for all $g,h\in G$ and for all $\sigma\in\{L,R\}$ satisfying
	\begin{align}
		\Ad{W_g}\circ\pi_0&=\pi_0\circ\alpha_0\circ\Theta\circ\eta_g\circ\beta_g^U\circ\Theta^{-1}\circ\alpha_0^{-1}\\
		\Ad{u_\sigma(g,h)}\circ\pi_\sigma&=\pi_\sigma\circ\alpha_\sigma\circ\eta_g^\sigma\circ\beta_g^{\sigma R}\circ\eta_h^{\sigma U}\circ(\beta_g^{\sigma R})^{-1}\circ(\eta^\sigma_{gh})^{-1}\circ\alpha_\sigma\\
		u_L(g,h)\otimes u_R(g,h)&=W_gW_hW_{gh}^{-1}.
	\end{align}
\end{lemma}
\begin{proof}
	See \cite{ogata2021h3gmathbb}.
\end{proof}
These objects will be the starting point for our definition. Now take $\tau\in\Aut{\AA}$ to be the automorphism that vertically translates every element of the $C^*$ algebra by one site upwards. We now define the translation action on the GNS space of $\omega_0$:
\begin{lemma}
	There exists a unique $v\in\UU(\HH_0)$ such that
	\begin{align}
	\Ad{v}\circ\pi_0&=\pi_0\circ\alpha_0\circ\Theta\circ\tau\circ\Theta^{-1}\circ\alpha_0^{-1}&&\text{and}&\pi_0(V_1)v\pi_0(V_1^\dagger)\Omega_0&=\Omega_0.
	\end{align}
\end{lemma}
\begin{proof}
	This is very straightforward.
\end{proof}
\section{Definition of the index}
We now have to define one more object
\begin{lemma}\label{lem:Definition_K}
	There exist $K_g^L\in\UU(\HH_L)$ and $K_g^R\in\UU(\HH_R)$ such that
	\begin{equation}
		v^\dagger W_g v W_g^\dagger=K_g^L\otimes K_g^R=K_g
	\end{equation}
	and they are unique (up to a phase). They satisfy the identity
	\begin{equation}
		\Ad{K^R_g}\circ\pi_0=\pi_0\circ \alpha_0\circ \tau^{-1}\circ \eta_g^R\circ\beta_g^{RU}\circ\tau\circ(\beta_g^{RU})^{-1}\circ(\eta_g^{R})^{-1}\circ\alpha_0^{-1}.\quad\footnote{Observe in particular that $\Theta$ has cancelled out}
	\end{equation}
\end{lemma}
\begin{proof}
	The only nontrivial part (that is not just some straightforward calculation) is showing that indeed
	\begin{equation}
		\omega_0\circ\alpha_0\circ \tau^{-1}\circ \eta_g^R\circ\beta_g^{RU}\circ\tau\circ(\beta_g^{RU})^{-1}\circ(\eta_g^{R})^{-1}\circ\alpha_0^{-1}\qe \omega_0.
	\end{equation}
	{\color{red}This I still have to do.}
\end{proof}
\begin{lemma}\label{lem:AdjointOverConeIsInCone}
	Take $\gamma^{\sigma}\in\Aut{(\AA_{C_\theta}\cup\tau(\AA_{C_\theta}))\cap\AA_{\sigma}}$ and take $\Gamma^{\sigma}$ (for all $\sigma\in\{L,R\}$) to be such that
	\begin{equation}
		\pi_{\sigma}\circ\alpha_{\sigma}\circ\gamma^{\sigma}\circ\alpha_{\sigma}^{-1}= \Ad{\Gamma^{\sigma}}\circ\pi_{\sigma}
	\end{equation}
	then
	\begin{equation}
		\Ad{W_g}(\Gamma_{L}\otimes\Gamma_{R})=\Ad{W_g}(\Gamma_L)\otimes\Ad{W_g}(\Gamma_{R}).
	\end{equation}
	Moreover there exist $\tilde\gamma^{\sigma}_g\in\Aut{(\AA_{C_\theta}\cup\tau(\AA_{C_\theta}))\cap\AA_{\sigma}}$ such that
	\begin{equation}
		\pi_{\sigma}\circ\alpha_{\sigma}\circ\tilde\gamma^{\sigma}_g\circ\alpha_{\sigma}^{-1}= \Ad{\Ad{W_g}(\Gamma^{\sigma})}\circ\pi_{\sigma}.
	\end{equation}
	If more generally we take $\Gamma^\sigma\in\pi_0\circ\alpha_0\circ\Theta(\AA_\sigma)''$ then $\Ad{W_g}(\Gamma^\sigma)\in\pi_0\circ\alpha_0\circ\Theta(\AA_\sigma)''$ $\forall g\in G.$
\end{lemma}
\begin{proof}
	content...
\end{proof}

\begin{lemma}\label{lem:Definition2Cochain}
	There exists a $C:G^2\rightarrow U(1)$ such that 
	\begin{equation}\label{eq:Definition2Cochain}
		K_g^R\Ad{W_g}\left(K_h^R\Ad{W_hW_{gh}^\dagger}\left((K_{gh}^R)^\dagger\right)\right)u_
		R(g,h)=C(g,h)v^\dagger u_R(g,h)v
	\end{equation}
	for all $g,h\in G.$
\end{lemma}
\begin{proof}
	Since the GNS representation is irreducible this is equivalent to showing that the left and righthandside of equation \eqref{eq:Definition2Cochain} have the same adjoint action on the GNS representation. We first prove the result for the full tensor product:
	\begin{align}
		&\Ad{v^\dagger u_L(g,h)v\otimes v^\dagger u_R(g,h) v}\circ\pi_0\\
		&=\Ad{v^\dagger (u_L(g,h)\otimes u_R(g,h)) v}\circ\pi_0\\
		&=\Ad{v^\dagger W_g W_h W_{gh}^{-1}v}\circ\pi_0\\
		&=\Ad{K_gW_g v^\dagger W_h W_{gh}^{-1}v}\circ\pi_0\\
		&=\Ad{K_gW_g K_h W_h v^\dagger W_{gh}^{-1}v}\circ\pi_0\\
		&=\Ad{K_gW_g K_h W_h W_{gh}^\dagger K_{gh}^\dagger }\circ\pi_0\\
		&=\Ad{K_gW_g K_h W_h W_{gh}^\dagger K_{gh}^\dagger W_{gh}W_h^\dagger W_g^\dagger W_gW_hW_{gh}^\dagger}\circ\pi_0\\
		&=\Ad{K_g\Ad{W_g}\left(K_h\Ad{W_hW_{gh}^\dagger}\left(K_{gh}^\dagger\right)\right)u_L(g,h)\otimes u_
			R(g,h)}\circ\pi_0.
	\end{align}
	Using lemma \ref{lem:AdjointOverConeIsInCone} we get that
	\begin{align}
		&(K_g^L\otimes K_g^R)\Ad{W_g}\left((K_h^L\otimes K_h^R)\Ad{W_hW_{gh}^\dagger}\left((K_{gh}^L\otimes K_{gh}^R)^{-1}\right)\right)\\
		&=K_g^L\Ad{W_g}\left(K_h^L\Ad{W_hW_{gh}^\dagger}\left((K_{gh}^L)^\dagger\right)\right)\otimes K_g^R\Ad{W_g}\left(K_h^R\Ad{W_hW_{gh}^\dagger}\left((K_{gh}^R)^\dagger\right)\right)
	\end{align}
	concluding the proof.
\end{proof}
\begin{lemma}
	The function $C$ as defined in lemma \ref{lem:Definition2Cochain} is a 2-cochain.
\end{lemma}
\begin{proof}
	Take lemma 2.4 from \cite{ogata2021h3gmathbb} there it is stated that there exists a 3-cochain $C'$ such that
	\begin{equation}\label{eq:defintion3CochainProof2Cochain}
		u_R(g,h)u_R(gh,k)u_R(g,hk)^\dagger\Ad{W_g}(u_R(h,k)^\dagger)=C'(g,h,k)\mathbbm{1}.
	\end{equation}
	It is clear that this 3-cochain is invariant under the substitution
	\begin{align}
		W_g&\rightarrow v^\dagger W_g v&u_R(g,h)&\rightarrow v^\dagger u_R(g,h)v.
	\end{align}
	If we now prove that it is also invariant under the substitution
	\begin{align}\label{eq:SubstitutionForProofCochain}
		W_g&\rightarrow K_g W_g&u_R(g,h)&\rightarrow K_g^R\Ad{W_g}\left(K_h^R\Ad{W_hW_{gh}^\dagger}\left((K_{gh}^R)^\dagger\right)\right)u_
		R(g,h)
	\end{align}
	we have proved our result because using equation \eqref{eq:Definition2Cochain} we then get that
	\begin{equation}
		\stkout{C'(g,h,k)}=C(h,k)C(gh,k)^{-1}C(g,hk)C(g,h)^{-1}\stkout{C'(g,h,k)}
	\end{equation}
	proving the result. Inserting substitution \eqref{eq:SubstitutionForProofCochain} into \eqref{eq:defintion3CochainProof2Cochain} gives
	\begin{align}
		&u_R(g,h)u_R(gh,k)u_R(g,hk)^\dagger\Ad{W_g}\left(u_R(h,k)^\dagger\right)\\
		\rightarrow&K_g^R\Ad{W_g}\left(K_h^R\Ad{W_hW_{gh}^\dagger}\left((K_{gh}^R)^\dagger\right)\right)u_
		R(g,h)\\
		\nonumber
		&K_{gh}^R\Ad{W_{gh}}\left(K_k^R\Ad{W_kW_{ghk}^\dagger}\left((K_{ghk}^R)^\dagger\right)\right)u_
		R(gh,k)\\
		\nonumber
		&u_R(g,hk)^\dagger \Ad{W_g}\left( \Ad{W_{hk}W_{ghk}^\dagger}(K^R_{ghk})(K^R_{hk})^{\dagger} \right)(K^R_g)^\dagger\\
		\nonumber
		&\Ad{K_gW_g}\left(u_R(h,k)^\dagger \Ad{W_h}\left(\Ad{W_kW_{hk}^\dagger}\left(K_{hk}^R\right)(K^R_k)^\dagger\right)(K^R_h)^\dagger\right).
	\end{align}
	Using the fact that $W_gW_hW_{gh}^\dagger=u_L\otimes u_R(g,h)$ one now gets
	\begin{align}
		=&K_g^R\Ad{W_g}\left(K_h^R\Ad{W_hW_{gh}^\dagger}\left((K_{gh}^R)^\dagger\right)\right)W_gW_hW_{gh}^\dagger u_L(g,h)^\dagger\\
		\nonumber
		&K_{gh}^R\Ad{W_{gh}}\left(K_k^R\Ad{W_kW_{ghk}^\dagger}\left((K_{ghk}^R)^\dagger\right)\right) W_{gh}W_kW_{ghk}^\dagger u_L(gh,k)^\dagger\\
		\nonumber
		&W_{ghk}W_{hk}^\dagger W_{g}^\dagger u_L(g,hk) \Ad{W_g}\left( \Ad{W_{hk}W_{ghk}^\dagger}(K^R_{ghk})(K^R_{hk})^{\dagger} \right)\stkout{(K^R_g)^\dagger}\\
		\nonumber
		&\:\stkout{K^R_g}\Ad{W_g}\left(u_R(h,k)^\dagger \Ad{W_h}\left(\Ad{W_kW_{hk}^\dagger}\left(K_{hk}^R\right)(K^R_k)^\dagger\right)(K^R_h)^\dagger\right)(K^R_g)^\dagger.
	\end{align}
	We will now use the fact that the $u_L$ commutes with everything that has support only on the right. Combining this with lemma \ref{lem:AdjointOverConeIsInCone} gives
	\begin{align}
		=&K_g^R\Ad{W_g}\left(K_h^R\Ad{W_hW_{gh}^\dagger}\left((K_{gh}^R)^\dagger\right)\right)W_gW_hW_{gh}^\dagger u_L(g,h)^\dagger\\
		\nonumber
		&K_{gh}^R\Ad{W_{gh}}\left(K_k^R\Ad{W_kW_{ghk}^\dagger}\left((K_{ghk}^R)^\dagger\right)\right) W_{gh}W_kW_{ghk}^\dagger u_L(gh,k)^\dagger\\
		\nonumber
		&\Ad{W_{ghk}W_{hk}^\dagger W_{g}^\dagger}  \left(\Ad{W_g}\left( \Ad{W_{hk}W_{ghk}^\dagger}(K^R_{ghk})(K^R_{hk})^{\dagger} \right)\right)\\
		\nonumber
		&u_R(g,hk)^\dagger\Ad{W_g}\left(u_R(h,k)^\dagger \Ad{W_h}\left(\Ad{W_kW_{hk}^\dagger}\left(K_{hk}^R\right)(K^R_k)^\dagger\right)(K^R_h)^\dagger\right)(K^R_g)^\dagger\\
		=&K_g^R\Ad{W_g}\left(K_h^R\Ad{W_hW_{gh}^\dagger}\left((K_{gh}^R)^\dagger\right)\right)W_gW_hW_{gh}^\dagger u_L(g,h)^\dagger\\
		\nonumber
		&K_{gh}^R\Ad{W_{gh}}\left(K_k^R\Ad{W_kW_{ghk}^\dagger}\left((K_{ghk}^R)^\dagger\right)\right)  \\
		\nonumber
		&\Ad{W_{gh}W_kW_{hk}^\dagger W_{g}^\dagger}  \left(\Ad{W_g}\left( \Ad{W_{hk}W_{ghk}^\dagger}(K^R_{ghk})(K^R_{hk})^{\dagger} \right)\right)\\
		\nonumber
		&u_R(gh,k)u_R(g,hk)^\dagger\Ad{W_g}\left(u_R(h,k)^\dagger \Ad{W_h}\left(\Ad{W_kW_{hk}^\dagger}\left(K_{hk}^R\right)(K^R_k)^\dagger\right)(K^R_h)^\dagger\right)(K^R_g)^\dagger\\
		=&K_g^R\Ad{W_g}\left(K_h^R\Ad{W_hW_{gh}^\dagger}\left((K_{gh}^R)^\dagger\right)\right)W_gW_hW_{gh}^\dagger (K^R_g)^\dagger\\
		\nonumber
		&K_{gh}^R\Ad{W_{gh}}\left(K_k^R\Ad{W_kW_{ghk}^\dagger}\left((K_{ghk}^R)^\dagger\right)\right)\\
		\nonumber
		&\Ad{W_{gh}W_kW_{hk}^\dagger W_{g}^\dagger}  \left(\Ad{W_g}\left( \Ad{W_{hk}W_{ghk}^\dagger}(K^R_{ghk})(K^R_{hk})^{\dagger} \right)\right)\\
		\nonumber
		&W_{gh}W_h^\dagger W_g^\dagger u_R(g,h) u_R(gh,k)u_R(g,hk)^\dagger\Ad{W_g}\left(u_R(h,k)^\dagger\right)\\
		\nonumber
		&\Ad{W_g}\left(\Ad{W_h}\left(\Ad{W_kW_{hk}^\dagger}\left(K_{hk}^R\right)(K^R_k)^\dagger\right)(K^R_h)^\dagger\right)(K^R_g)^\dagger
	\end{align}
	Filling in the equation for the 3-cochain (equation \eqref{eq:defintion3CochainProof2Cochain}) now gives
	\begin{align}
		=&C'(g,h,k)K_g^R\Ad{W_g}\left(K_h^R\Ad{W_hW_{gh}^\dagger}\left((K_{gh}^R)^\dagger\right)\right)W_gW_hW_{gh}^\dagger \\
		\nonumber
		&K_{gh}^R\Ad{W_{gh}}\left(K_k^R\Ad{W_kW_{ghk}^\dagger}\left((K_{ghk}^R)^\dagger\right)\right)\\
		\nonumber
		&\Ad{W_{gh}W_kW_{hk}^\dagger \stkout{W_{g}^\dagger}}  \left(\stkout{\Ad{W_g}}\left( \Ad{W_{hk}W_{ghk}^\dagger}(K^R_{ghk})(K^R_{hk})^{\dagger} \right)\right)\\
		\nonumber
		&W_{gh}W_h^\dagger W_g^\dagger\Ad{W_g}\left(\Ad{W_h}\left(\Ad{W_kW_{hk}^\dagger}\left(K_{hk}^R\right)(K^R_k)^\dagger\right)(K^R_h)^\dagger\right)(K^R_g)^\dagger.
	\end{align}
	Fully writing out the adjoints now gives:
	\begin{align}
	=&C'(g,h,k)K_g^RW_gK_h^RW_hW_{gh}^\dagger (K_{gh}^R)^\dagger W_{gh}W_h^\dagger W_g^\dagger W_gW_hW_{gh}^\dagger \\
	\nonumber
	&K_{gh}^RW_{gh}K_k^RW_kW_{ghk}^\dagger(K_{ghk}^R)^\dagger W_{ghk}W_k^\dagger W_{gh}^\dagger  \\
	\nonumber
	&W_{gh}W_kW_{hk}^\dagger W_{hk}W_{ghk}^\dagger K^R_{ghk} W_{ghk}W_{hk}^\dagger(K^R_{hk})^{\dagger} W_{hk}W_k^\dagger W_{gh}^\dagger\\
	\nonumber
	&W_{gh}W_h^\dagger W_g^\dagger W_gW_h W_kW_{hk}^\dagger K_{hk}^RW_{hk}W_k^\dagger(K^R_k)^\dagger W_h^\dagger (K^R_h)^\dagger W_g^\dagger (K^R_g)^\dagger\\
	=&\mathbbm{1}C'(g,h,k)
	\end{align}
	concluding the proof.
\end{proof}
Our translation index is now defined as
\begin{definition}
	Take $C$ to be the 2-cochain defined \ref{lem:Definition2Cochain} then we define the index as
	\begin{equation}
	\textrm{Index}(\theta,\tilde{\beta}_g,\eta_g,\alpha_0,\Theta,\omega,\omega_0)\defeq\expval{C}\in H^2(G,\TT)
	\end{equation}
	and (as advertised) it is only a function of the automorphisms (and the product state) not on the choice of the GNS triple of $\omega_0$ or on the choice of phases in $W_g,u_L(g,h),u_R(g,h),v,K_g^L$ and $K_g^R$.
\end{definition}
\begin{proof}
	Clearly the construction is invariant under the choice of GNS triple. Now we will show that it is invariant under the choice of phases of our operators. Clearly the 2-cochain $C$ is invariant under
	\begin{align}
		u_L(g,h)&\rightarrow \alpha(g)\alpha(g)\alpha(gh)^{-1}\beta(g,h)^{-1} u_L(g,h)&u_R(g,h)&\rightarrow \beta(g,h)u_R(g,h)\\
		W_g&\rightarrow\alpha(g)W_g&v&\rightarrow \gamma v.
	\end{align}
	Under the transformation
	\begin{align}
		K_g^L&\rightarrow \delta(g)^{-1}K_g^L&K_g^R&\rightarrow \delta(g)K_g^R
	\end{align}
	we get $C(g,h)\rightarrow \delta(g)\delta(h)\delta(gh)^{-1}C(g,h)$ which is clearly still in the same equivalence class concluding the proof.
\end{proof}
\section{The index is independent of the choices we made}
In this section we will show that the index is only dependent on $\omega$ and $\omega_0$ and not on the choices of our automorphisms. First we show independence of $\alpha$ and its decomposition.
\begin{lemma}
	Take $\omega_{01},\omega_{02}\in\PP(\AA)$ product states, take $\alpha_1,\alpha_2\in\Aut{\AA}$, take $V_{11},V_{12}\in\UU(\AA)$, take $\alpha_{L/R,1},\alpha_{L/R,2}\in\Aut{\AA_{L/R}}$ and take $\Theta_1,\Theta_2\in \Aut{\AA_{(C_\theta\cup \tau(C_\theta))^c}}$ to be such that
	\begin{align}
		\alpha_1&=\Ad{V_{11}}\circ\alpha_{01}\circ\Theta_1&\alpha_2&=\Ad{V_{12}}\circ\alpha_{02}\circ \Theta_2
	\end{align}
	and satisfying
	\begin{equation}
		\omega_{01}\circ\alpha_1=\omega_{02}\circ\alpha_2=\omega
	\end{equation}
	then
	\begin{equation}
		\textrm{Index}(\theta,\tilde{\beta}_g,\eta_g,\alpha_{0,1},\Theta_1,\omega,\omega_{01})=\textrm{Index}(\theta,\tilde{\beta}_g,\eta_g,\alpha_{0,2},\Theta_2,\omega,\omega_{02}).
	\end{equation}
\end{lemma}
\begin{proof}
	We will first prove the result in the case that $\omega_0=\omega_{01}=\omega_{02}$ and then generalise this result. Since $\omega_0\circ\alpha_2\circ\alpha_1^{-1}=\omega_0$ there exists a $\tilde{w}\in\UU(\HH_0)$ such that
	\begin{equation}
		\pi_0\circ\alpha_2\circ\alpha_1^{-1}=\Ad{\tilde{w}}\circ\pi_0.
	\end{equation}
	Now define $w\in\UU(\HH_0)$ to be
	\begin{equation}
		w\defeq \pi_0(V_{11})\tilde{w} \pi_0(V_{11}^\dagger)
	\end{equation}
	then
	\begin{equation}
		\pi_0\circ\alpha_{02}\circ\Theta_2=\Ad{w}\circ\pi_0\circ\alpha_{01}\circ\Theta_1.
	\end{equation}
	Now take $W_{g,1},u_{R,1}(g,h)$ and $K^R_{g,1}$ to be the operators belonging to the first choice (with arbitrary phases). 
	We have (see \cite{ogata2021h3gmathbb} lemma 2.11)
	\begin{align}
		\Ad{wW_{g,1}w^\dagger}\circ\pi_0&=\pi_0\circ \alpha_{02}\circ\Theta_2\circ\eta_g\circ\beta_g^U\circ\Theta_2^{-1}\circ\alpha_{02}^{-1},\\
		\Ad{wu_{R,1}(g,h)w^\dagger}\circ\pi_0&=\pi_0\circ \alpha_{02}\circ\eta_g^R\circ\beta_g^{RU}\eta_h^R\beta_{h}^{RU}(\beta_{gh}^{RU})^{-1}(\eta_{gh}^R)^{-1}\circ\alpha_{02}^{-1}
	\end{align}
	and through similar arguments we get
	\begin{align}
		\Ad{wvw^\dagger}\circ\pi_0&=\pi_0\circ\alpha_{02}\circ\Theta_2\circ\tau\circ\Theta_2^{-1}\circ\alpha_{02}^{-1}\\
		\Ad{wK^R_{g,1}w^\dagger}\circ\pi_0&=\pi_0\circ \alpha_{02}\circ\tau^{-1}\circ\eta_g^R\circ\beta_g^{RU}\circ\tau\circ(\beta_g^{RU})^{-1}\circ(\eta^R_g)^{-1}\circ\alpha_{02}^{-1}.
	\end{align}
	This shows that $wW_{g,1}w^\dagger,wu_{R,1}(g,h)w^\dagger$ and $wK^R_{g,1}w^\dagger$ are operators belonging to the second choice (with our new translation operator). Since our index is invariant under this substitution this concludes the proof when $\omega_0=\omega_{01}=\omega_{02}$. Now suppose that $\omega_{01}\neq\omega_{02}$. Since they are both product states there exists a $\gamma\in\Aut{\AA}$ satisfying $\omega_{02}=\omega_{01}\circ\gamma$ that is of the form $\gamma=\gamma^L\otimes\gamma^R$. We now have
	\begin{equation}
		\textrm{Index}(\theta,\tilde{\beta}_g,\eta_g,\alpha_{0,2},\Theta_2,\omega,\omega_{02})=\textrm{Index}(\theta,\tilde{\beta}_g,\eta_g,\alpha_{0,2},\Theta_2,\omega,\omega_{01}\circ\gamma)=\textrm{Index}(\theta,\tilde{\beta}_g,\eta_g,\gamma\circ\alpha_{0,2},\Theta_2,\omega,\omega_{01})
	\end{equation}
	concluding the proof.
\end{proof}
We now define a subgroup of $\UU(\pi_0\circ\alpha_0\circ\Theta(\AA)'')$ that is includes both the group $\UU(\pi_0\circ\alpha_0\circ\Theta(\AA_R))$ and the cone automorphisms:
\begin{definition}
	Take $u\in\UU(\pi_0\circ\alpha_0\circ\Theta(\AA)'')$ then we say that $u$ is inner after cone on the right (or in short $u\in \textrm{IAC}_R(\alpha_0,\Theta)$) if there exists an $A\in\AA_R$ and some $\xi\in\Aut{\AA_{C_\theta}\cap\AA_R}$ such that
	\begin{equation}
	\Ad{u}\circ\pi_0\circ\alpha_0\circ\Theta=\pi_0\circ\alpha_0\circ\Theta\circ\Ad{A}\circ\xi.
	\end{equation}
	$\textrm{IAC}_R$ is a subgroup of $\UU(\pi_0\circ\alpha_0\circ\Theta(\AA)'')$. If we define similarly $\textrm{IAC}_R$ we get that
	\begin{align}\label{eq:CommutantProperty}
	\textrm{IAC}_R(\alpha_0,\Theta)&\subseteq\textrm{IAC}_L(\alpha_0,\Theta)'&\textrm{IAC}_L(\alpha_0,\Theta)&\subseteq\textrm{IAC}_R(\alpha_0,\Theta)'.
	\end{align}
\end{definition}
\begin{proof}
	content...
\end{proof}
We now show that the index is invariant under some transformation parametrised by the group $\textrm{IAC}_R$:
\begin{lemma}\label{lem:TransformationUnderDelta}
	Take $\omega_0,\theta,\alpha_0,\Theta$ and $\eta_g$ as usual, take $v,W_{g,1},u_{\sigma,1}(g,h)$ and $K_{g,1}^\sigma$ the operators corresponding to these automorphisms and take $\delta^\sigma_g\in\textrm{IAC}_R(\alpha_0,\Theta)$. Define
	\begin{align}
		W_{g,2}&\defeq\delta_g W_{g,1}\\
		u_{R,2}(g,h)&\defeq \delta_g^R W_{g,1}\delta_h^R W_{g,1}^\dagger u_{R,1}(g,h)(\delta_{gh}^R)^\dagger\\
		K^R_{g,2}&\defeq v^\dagger \delta_g^R v K_{g,1}^R (\delta_g^R)^\dagger
	\end{align}
	then the index of $v,W_{g,1},u_{\sigma,1}(g,h)$ and $K_{g,1}^\sigma$ is equal to the index of $v,W_{g,2},u_{\sigma,2}(g,h)$ and $K_{g,2}^\sigma$.
\end{lemma}
\begin{proof}
	We have
	\begin{align}
		&K_{g,2}^RW_{g,2}K_{h,2}^{R}W_{h,2}W_{gh,2}^\dagger(K_{gh,2}^R)^\dagger W_{gh,2}W_{h,2}^\dagger W_{g,2}^\dagger u_{R,2}(g,h)\\
		=&K_{g,2}^RW_{g,2}K_{h,2}^{R}W_{h,2}W_{gh,2}^\dagger(K_{gh,2}^R)^\dagger u_{L,2}(g,h)^\dagger\\
		=&K_{g,2}^RW_{g,2}K_{h,2}^{R}W_{h,2}W_{gh,2}^\dagger u_{L,2}(g,h)^\dagger (K_{gh,2}^R)^\dagger\\
		=&K_{g,2}^RW_{g,2}K_{h,2}^{R}\stkout{W_{h,2}W_{gh,2}^\dagger W_{gh,2}W_{h,2}^\dagger} W_{g,2}^\dagger u_{R,2}(g,h) (K_{gh,2}^R)^\dagger.
	\end{align}
	Filling this in the definition of the 2-cochain gives
	\begin{align}
		C_2(g,h)&=K_{g,2}^RW_{g,2}K_{h,2}^{R}W_{g,2}^\dagger u_{R,2}(g,h) (K_{gh,2}^R)^\dagger v^\dagger u_{R,2}(g,h)^\dagger v\\
		&=v^\dagger \delta_g^R v K_{g,1}^R (\delta_g^R)^\dagger \delta_g W_{g,1}v^\dagger \delta_h^R v K_{h,1}^R (\delta_h^R)^\dagger W_{g,1}^\dagger \delta_g^\dagger\\
		\nonumber
		&\qquad \delta_g^R W_{g,1}\delta_h^R W_{g,1}^\dagger u_{R,1}(g,h)\stkout{(\delta_{gh}^R)^\dagger \delta_{gh}^R} (K_{gh,1}^R)^\dagger v^\dagger \stkout{(\delta_{gh}^R)^\dagger v  v^\dagger \delta^R_{gh}} u_{R,1}(g,h)^\dagger W_{g,1}(\delta^R_h)^\dagger W_{g,1}^\dagger(\delta_g^R)^\dagger v.
	\end{align}
	We now insert $(K_{h,1}^R)^\dagger W_{g,1}^\dagger (K_{g,1}^R)^\dagger K_{g,1}^R W_{g,1}K_{h,1}^R=\mathbbm{1}:$
	\begin{align}	
		C_2(g,h)&=v^\dagger \delta_g^R v K_{g,1}^R (\delta_g^R)^\dagger \delta_g W_{g,1}v^\dagger \delta_h^R v K_{h,1}^R (\delta_h^R)^\dagger W_{g,1}^\dagger \delta_g^\dagger\\
		\nonumber
		&\qquad \delta_g^R W_{g,1}\delta_h^R \left((K_{h,1}^R)^\dagger W_{g,1}^\dagger (K_{g,1}^R)^\dagger K_{g,1}^R W_{g,1}K_{h,1}^R\right) W_{g,1}^\dagger u_{R,1}(g,h) (K_{gh,1}^R)^\dagger v^\dagger u_{R,1}(g,h)^\dagger v\\
		\nonumber
		&\qquad v^\dagger W_{g,1}(\delta^R_h)^\dagger W_{g,1}^\dagger(\delta_g^R)^\dagger v.
	\end{align}
	Filling in the old index now gives:
	\begin{align}
		C_2(g,h)&=C_1(g,h)v^\dagger \delta_g^R v K_{g,1}^R (\delta_g^R)^\dagger \delta_g W_{g,1}v^\dagger \delta_h^R v K_{h,1}^R (\delta_h^R)^\dagger W_{g,1}^\dagger \delta_g^\dagger\\
		\nonumber
		&\qquad \delta_g^R W_{g,1}\delta_h^R (K_{h,1}^R)^\dagger W_{g,1}^\dagger (K_{g,1}^R)^\dagger v^\dagger W_{g,1}(\delta^R_h)^\dagger W_{g,1}^\dagger(\delta_g^R)^\dagger v
	\end{align}
	Using equation \eqref{eq:CommutantProperty} we now get:
	\begin{align}
		C_2(g,h)&=C_1(g,h)v^\dagger \delta_g^R v K_{g,1}^R \stkout{\delta_g^L} W_{g,1}v^\dagger \delta_h^R v K_{h,1}^R (\delta_h^R)^\dagger W_{g,1}^\dagger \stkout{(\delta_g^L)^\dagger}\\
		\nonumber
		&\qquad  W_{g,1}\delta_h^R (K_{h,1}^R)^\dagger W_{g,1}^\dagger (K_{g,1}^R)^\dagger v^\dagger W_{g,1}(\delta^R_h)^\dagger W_{g,1}^\dagger(\delta_g^R)^\dagger v\\
		&=C_1(g,h)v^\dagger \delta_g^R v K_{g,1}^\stkout{R} W_{g,1}v^\dagger \delta_h^R v \stkout{K_{h,1}^R (\delta_h^R)^\dagger W_{g,1}^\dagger W_{g,1}\delta_h^R (K_{h,1}^R)^\dagger} W_{g,1}^\dagger (K_{g,1}^\stkout{R})^\dagger v^\dagger W_{g,1}(\delta^R_h)^\dagger W_{g,1}^\dagger(\delta_g^R)^\dagger v\\
		&=C_1(g,h)v^\dagger \delta_g^R v K_{g,1} W_{g,1}v^\dagger \delta_h^R v W_{g,1}^\dagger (K_{g,1})^\dagger v^\dagger W_{g,1}(\delta^R_h)^\dagger W_{g,1}^\dagger(\delta_g^R)^\dagger v\\
		&=C_1(g,h)v^\dagger \delta_g^R W_{g,1} \delta_h^R W_{g,1}^\dagger  W_{g,1}(\delta^R_h)^\dagger W_{g,1}^\dagger(\delta_g^R)^\dagger v\\
		&=C_1(g,h)
	\end{align}
	concluding the proof.
\end{proof}
We will now show that the index is independent on the choice of $\tilde{\beta}_g$ and its decomposition.
\begin{lemma}\label{lem:InvarianceUnderChoiceBeta}
	Take $\tilde{\beta}_{g,1},\tilde{\beta}_{g,1}\in\Aut{\AA},V_{g,21},V_{g,22}\in\UU(\AA),\eta_{g,1}^L,\eta_{g,2}^L\in \Aut{\AA_L\cap\AA_{C_\theta}}$ and $\eta_{g,1}^R,\eta_{g,2}^R\in \Aut{\AA_R\cap\AA_{C_\theta}}$ such that there exist $V_{g,21},V_{g,22}\in\UU(\AA)$ satisfying
	\begin{align}
		\tilde{\beta}_{g,1}&=\Ad{V_{g,21}}\circ\eta_{g,1}\circ\beta^U_g&\tilde{\beta}_{g,2}&=\Ad{V_{g,22}}\circ\eta_{g,2}\circ\beta^U_g
	\end{align}
	and
	\begin{equation}
		\omega\circ\tilde{\beta}_{g,1}=\omega\circ\tilde{\beta}_{g,2}=\omega
	\end{equation}
	then
	\begin{equation}
		\textrm{Index}(\theta,\tilde{\beta}_{g,1},\eta_{g,1},\alpha_{0},\Theta,\omega,\omega_0)=\textrm{Index}(\theta,\tilde{\beta}_{g,2},\eta_{g,2},\alpha_{0},\Theta,\omega,\omega_0).
	\end{equation}
\end{lemma}
\begin{proof}
	Take 
	\begin{equation}
		\alpha=\Ad{V_1}\circ\alpha_{0}\circ\Theta
	\end{equation}
	the usual decomposition. Since
	\begin{equation}
		\omega_0\circ\alpha\circ\tilde{\beta}_{g,1}\circ(\tilde{\beta}_{g,2})^{-1}=\omega_0\circ\alpha
	\end{equation}
	there exist $\tilde{\delta}_g\in\UU(\HH_0)$ such that
	\begin{equation}
		\Ad{\tilde{\delta}_g}\circ\pi_0\circ\alpha=\pi_0\circ\alpha\circ\tilde{\beta}_{g,2}(\tilde{\beta}_{g,1})^{-1}.
	\end{equation}
	After a few calculations {\color{red}I still have to write these out} one gets that
	\begin{equation}
		\Ad{\pi_0\circ\alpha_0\circ\Theta(V_{g,22}^\dagger)\pi_0(V_1^\dagger)\tilde{\delta}_g\pi_0(V_1)\pi_0\circ\alpha_0\circ\Theta(V_{g,21})}\circ\pi_0=\pi_0\circ\alpha_0\circ\eta_{g,2}\circ(\eta_{g,1})^{-1}\circ\alpha_0^{-1}.
	\end{equation}
	Since the last equation is split we can take $\delta_g^L\in\UU(\HH_L)$ and $\delta_g^R\in\UU(\HH_R)$ such that
	\begin{equation}
		\delta_g^L\otimes\delta_g^R=\alpha_0\circ\Theta(V_{g,21}^\dagger)\pi_0(V_1^\dagger)\tilde{\delta}_g\pi_0(V_1)\pi_0\circ\alpha_0\circ\Theta(V_{g,22}).
	\end{equation}
	Take $W_{g,1},u_{R,1}(g,h)$ and $K^R_{g,1}$ to be the operators belonging to the first choice (with arbitrary phases). Define
	\begin{align}
		W_{g,2}&\defeq\delta_g W_{g,1}\\
		u_{R,2}(g,h)&\defeq \delta_g^R W_{g,1}\delta_h^R W_{g,1}^\dagger u_{R,1}(g,h)(\delta_{gh}^R)^\dagger\\
		K^R_{g,2}&\defeq v^\dagger \delta_g^R v K_{g,1}^R (\delta_g^R)^\dagger
	\end{align}
	then $W_{g,2},u_{R,2}(g,h)$ and $K^R_{g,2}$ are operators belonging to the second choice. Now using lemma \ref{lem:TransformationUnderDelta} concludes the proof.
\end{proof}
\begin{lemma}
	The index is independent of the choice of angle $\theta$.
\end{lemma}
\begin{proof}
	After what we've shown so far this proof is trivial.
\end{proof}
Due to all these considerations we will write the index as $\textrm{index}(\omega)$ from here on onward.
\section{Index is invariant under locally generated automorphisms}
Take $H:I\subset \ZZ^2\mapsto H(I)\in\AA_I\subset\AA_{\text{loc}}$ to be an interaction such that
\begin{align}
\beta_g(H(I))&=H(I)&\tau(H(I))&=H(I')
\end{align}
where $I'$ is just $I$ shifted upwards. Take $\gamma:\RR\rightarrow\Aut{\AA}$ to be the locally generated automorphism generated from this interaction. We will denote by $\gamma\defeq \gamma(1)$ and for some $I\subset \ZZ$ we will denote $\gamma_I$ to mean the locally generated automorphism generated by 
\begin{equation}
	H_I:J\subset\ZZ\mapsto\left\{\begin{matrix}H(J)&\text{if }J\subset I\\0&\text{otherwise.}\end{matrix}\right.
\end{equation}
We will first need a lemma of the form
\begin{lemma}\label{lem:TwoAngleLemma}
	Take $\theta_1$ and $\theta_2$ such that $0<\theta_1<\theta_2<\pi/2$. Then for all
	\begin{enumerate}
		\item $\Theta_1\in\Aut{\AA_{(C_{\theta_2}\cup \tau(C_{\theta_2}))^c}}$ there exists an $a_1\in\UU(\AA)$ and a $\Theta_2\in\Aut{\AA_{(C_{\theta_1}\cup \tau(C_{\theta_1}))^c}}$ such that
		\begin{equation}
		\gamma_{L}^{-1}\otimes\gamma_{L}^{-1}\circ\Theta_1\circ\gamma=\Ad{a_1}\circ\Theta_2.
		\end{equation}
		\item $\eta_{g}^{L/R}\in\Aut{\AA_{C_{\theta_1}}\cap\AA_{L/R}}$ there exist $a_2\in\UU(\AA),a_3\in\UU(\AA_R)$ such that
		\begin{align}
		\gamma^{-1}\circ\eta_{g}^L\otimes\eta_{g}^R\circ\gamma&=\Ad{a_2}\circ(\gamma_{C_\theta\cup L}^{-1}\circ \eta_{g,2}^L \circ\gamma_{C_\theta\cup L})\otimes(\gamma_{C_\theta\cup R}^{-1}\circ\eta_{g,2}^R\circ\gamma_{C_\theta\cup R})\\
		\gamma_\sigma^{-1}\eta_g^\sigma\gamma_\sigma&=\Ad{a_3}\circ \gamma_{C_\theta\cup \sigma}^{-1}\circ \eta_{g,2}^\sigma \circ\gamma_{C_\theta\cup \sigma}
		\end{align}
	\end{enumerate}
\end{lemma}
\begin{proof}
	This looks like it has a standard Lieb-Robinson type argument.
\end{proof}
Another lemma that will be necessary:
\begin{lemma}\label{lem:GroupActionInCone}
	Take $\theta\in]0,\pi/2[$ then there exists an $a\in\UU(\AA)$ and some $\xi_{g}^{L/R}\in\Aut{\AA_{C_{\theta}}\cup\AA_{L/R}}$ such that
	\begin{equation}
	\gamma^{-1}\circ\beta_g^U\circ\gamma\circ(\beta_g^U)^{-1}=\Ad{a}\circ\xi^L_g\otimes\xi^R_g.
	\end{equation}
\end{lemma}
\begin{proof}
	This also looks like it has a standard Lieb-Robinson type argument.
\end{proof}
\begin{theorem}
	$\textrm{index}(\omega)=\textrm{index}(\omega\circ\gamma)$
\end{theorem}
\begin{proof}
	Take $\alpha\in\QAut{\AA}$ such that $\omega=\omega_0\circ\alpha$ for some product state $\omega_0$. Take $0<\theta_1<\theta_2<\theta_3<\pi/2$ arbitrary. Take $\alpha_L\in\Aut{\AA_L},\alpha_R\in\Aut{\AA_R},V_1\in\UU(\AA)$ and $\Theta\in\Aut{\left(\AA_{C_{\theta_3}} \cup \tau(\AA_{C_{\theta_3}})\right)^c}$ such that
	\begin{equation}
	\alpha=\Ad{V_1}\circ\alpha_L\otimes\alpha_R\circ\Theta.
	\end{equation}
	Take $\tilde{\beta}_g$ such that $\omega\circ\tilde{\beta}_{g,1}=\omega$ and such that there exists some $V_{2,g}\in\UU(\AA)$ and some $\eta_g^{L/R}\in\Aut{\AA_{C_{\theta_1}}\cap\AA_{L/R}}$ satisfying
	\begin{equation}
		\tilde{\beta}_{g,1}=\Ad{V_{2,g}}\circ\eta_{g}^L\otimes\eta_{g}^R\circ\beta_g^U.
	\end{equation}
	Take $W_g,u_{L/R}(g,h),v$ and $K_g^{L/R}$ operators belonging to these automorphisms (see lemmas \ref{lem:Definition_W_And_u} and \ref{lem:Definition_K}). Clearly $\alpha\circ\gamma$ satisfies $\omega\circ\gamma=\omega_0\circ\alpha\circ\gamma$. We also have that $\alpha\circ\gamma\in\QAut{\AA}$. To show this notice that
	\begin{equation}
		\alpha\circ\gamma=\Ad{V_1}\circ(\alpha_L\circ\gamma_{H_L}\otimes\alpha_R\circ\gamma_{H_R})\circ\gamma_{H_L}^{-1}\otimes\gamma_{H_R}^{-1}\circ\Theta\circ\gamma.
	\end{equation}
	Because of lemma \ref{lem:TwoAngleLemma} part 1 there exists a $\tilde{\Theta}\in\Aut{\AA_{C_{\theta_2}}\cup\tau(\AA_{C_{\theta_2}})}$ and some $\tilde V_1,A_1\in\UU(\AA)$ such that
	\begin{align}
		\alpha\circ\gamma&=\Ad{\tilde V_1}\circ(\alpha_L\circ\gamma_{H_L}\otimes\alpha_R\circ\gamma_{H_R})\circ\tilde\Theta\\
		(\alpha_L\circ\gamma_{H_L}\otimes\alpha_R\circ\gamma_{H_R})\circ\tilde\Theta&=\Ad{A_1}\circ \alpha_L\otimes\alpha_R\circ\Theta\circ\gamma.
	\end{align}
	If we now define the automorphism $\tilde{\beta}_{g,2}=\gamma^{-1}_H\beta_{g,1}\gamma$ then this indeed satisfies that $\omega\circ\gamma\circ\tilde{\beta}_{g,2}=\omega\circ\gamma$. Define $\tilde\eta_g^\sigma\defeq\gamma_{C_{\theta_2}\cap\sigma}^{-1}\eta_g^\sigma\gamma_{C_{\theta_2}\cap\sigma}$ then by lemma \ref{lem:TwoAngleLemma} part 2 and lemma \ref{lem:GroupActionInCone} there exists a $\tilde V_{2,g}\in\UU(\AA),A_{2,g}^{\sigma}\in\UU(\AA_\sigma)$ such that
	\begin{align}
	\gamma^{-1}\tilde{\beta}_g\gamma&=\Ad{\tilde V_{2,g}}\circ\tilde{\eta}_g^L\otimes\tilde{\eta}_g^R\circ\beta_g^U\\
	\tilde{\eta}_g^\sigma&=\Ad{A_{2,g}^\sigma}\circ\gamma_{\sigma}^{-1}\circ\eta_g^\sigma\circ\beta_g^{\sigma U}\circ\gamma_{H_\sigma}\circ(\beta_g^{\sigma U})^{-1}.
	\end{align}
	Take
	\begin{equation}
	\phi_1:\textrm{IAC}_R(\alpha_0,\Theta) \rightarrow \textrm{IAC}_R(\alpha_0,\Theta\circ \gamma \circ \gamma_L^{-1}\otimes\gamma_R^{-1})
	\end{equation}
	the (I hope unique) group morphism such that for any $x\in\textrm{IAC}_R(\alpha_0,\Theta)$ satisfying that there exists a $\xi\in\Aut{\AA_{C_\theta}\cap\AA_R}$ and an $A\in\AA_R$ such that
	\begin{equation}
	\Ad{x}\circ\pi_0\circ\alpha_0\circ\Theta=\pi_0\circ\alpha_0\circ\Theta\circ\Ad{A}\circ\xi
	\end{equation}
	we have that
	\begin{equation}
		\Ad{\phi_1(x)}\circ\pi_0\circ\alpha_0\circ\Theta=\pi_0\circ\alpha_0\circ\Theta\circ\gamma\circ\gamma_{R}^{-1}\circ\Ad{A}\circ\xi\circ\gamma_{R}\circ\gamma^{-1}.
	\end{equation}
	Take (similarly)
	\begin{equation}
	\phi_2:\{W_g|g\in G\}\rightarrow \UU(\HH)
	\end{equation}
	satisfying
	\begin{equation}
	\Ad{\phi_2(W_g)}\circ\pi_0=\pi_0\circ \alpha_0\circ\Theta\circ\gamma\circ(\gamma_L^{-1}\otimes\gamma_R^{-1})\circ\eta_g\circ\beta_g^U\circ(\gamma_L\otimes\gamma_R)\circ\gamma^{-1}\circ\Theta^{-1}\circ\alpha_0^{-1}
	\end{equation}
	with some arbitrary $G-$dependent phase. {\color{red}Of the following I am not sure this is probably related to the uniqueness question above.} We have
	\begin{align}
	\phi_1(\Ad{W_g}(x))&=\Ad{\phi_2(W_g)}(\phi_1(x))\\
	\phi_1(\Ad{v}(x))&=\Ad{v}(\phi_1(x)).
	\end{align}
	The following operators now belong to $\omega\circ\gamma:$
	\begin{align}
		\tilde{v}&=\pi_0(A_1)v\pi_0(A_1)^\dagger\\
		\tilde{W}_g&=\pi_0(A_1)\phi_1(\delta^L_g\otimes\delta^R_g)\phi_2( W_g)\pi_0(A_1^\dagger)\\
		\tilde u_\sigma(g,h)&=\pi_0(A_1) \phi_1\left(\delta^\sigma_g W_g\delta^\sigma_h W_g^\dagger u_\sigma(g,h)(\delta^\sigma_{gh})^\dagger\right)\pi_0(A_1)^\dagger
	\end{align}
	where $\delta^\sigma_g=\pi_0\circ\alpha_0\circ\Theta\circ\gamma_{H_\sigma}(A^\sigma_{2,g}).$ The fact that the index is invariant under $\Ad{\pi_0(A_1)}$ and under $\phi(\cdot)$ should be rather clear. To show that it is invariant under the transformation with de $\delta^\sigma_g$ we invoke lemma \ref{lem:TransformationUnderDelta}.
\end{proof}
\appendix
\section{General statements about GNS representations}
For this section take $\omega\in\PP(\AA)$ and $(\HH,\pi,\Omega)$ a GNS triple for $\omega$. We will prove some general properties for this GNS triple.
\begin{lemma}
	Take $I\subset\ZZ$ and take $\xi\in\Aut{\AA_I}$ such that $\omega\circ\xi\qe \omega$ then there exists an $x\in\UU(\pi(\AA_I)'')$ such that
	\begin{equation}
		\pi\circ\xi=\Ad{x}\circ\pi.
	\end{equation}
\end{lemma}
\begin{proof}
	{\color{red}I don't explicitly use this in my proofs but it would certainly be a nice to have as it would make make things easier...}
\end{proof}
\begin{lemma}
	Take $x\in\pi(\AA_I)''$ and $y\in\pi(\AA_J)''$ for $I,J\subset\ZZ$ nonoverlapping then
	\begin{equation}
	[x,y]=0.
	\end{equation}
\end{lemma}
\begin{proof}
	Take $(X_n)_{n\in\NN}$ a sequence in $\AA_I$ converging to $x$ and $(Y_n)_{n\in\NN}$ a sequence in $\AA_J$ converging to $y$ both in weak operator topology. Take $A,B\in\AA$ arbitrary then
	\begin{equation}
	\bra{\Omega}\pi(A)^\dagger [x,y]\pi(B)\ket{\Omega}=\lim_{m\rightarrow\infty}\lim_{n\rightarrow \infty}\bra{\Omega}\pi(A)^\dagger [X_n,Y_m]\pi(B)\ket{\Omega}=0
	\end{equation}
	concluding the proof.
\end{proof}
\bibliography{TSPT}
\bibliographystyle{plain}
\end{document}