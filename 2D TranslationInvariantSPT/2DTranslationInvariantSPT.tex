\documentclass[12pt,a4paper,twoside]{article}
\usepackage{graphicx,xcolor,textpos}
\usepackage{helvet}
\usepackage[english]{babel}
\usepackage{amsmath}
\usepackage{amsthm}
\usepackage{bbm}
\usepackage{amssymb}
\usepackage{hyperref}
\usepackage{relsize}
\usepackage[margin=0.7in]{geometry}
\usepackage{physics}
\usepackage{enumitem}
\usepackage{mathtools}
\usepackage{changepage}
\usepackage{caption}
\usepackage{subcaption}
\usepackage{verbatim}
\usepackage{url}
\usepackage{tikz}
\usetikzlibrary{calc}

\renewcommand{\d}{\text{d}}
\renewcommand{\O}{\mathcal{O}}
\newcommand{\e}{\mathlarger{e}}
\newcommand{\defeq}{\vcentcolon=}
\let\originalleft\left
\let\originalright\right
\renewcommand{\left}{\mathopen{}\mathclose\bgroup\originalleft}
\renewcommand{\right}{\aftergroup\egroup\originalright}
\title{1D SPT classification with translation symmetry}
\author{Tijl Jappens}
\date{\today}

\newcommand{\UU}{\mathcal U}
\newcommand{\KK}{\mathcal K}
\newcommand{\BB}{\mathcal B}
\newcommand{\PP}{\mathcal P}
\newcommand{\HH}{\mathcal H}
\newcommand{\ZZ}{\mathbb Z}
\newcommand{\CC}{\mathbb C}
\newcommand{\TT}{\mathbb T}
\renewcommand{\AA}{\mathcal A}
\newcommand{\LL}{\mathcal L}
\newcommand{\RR}{\mathbb R}
\newcommand{\NN}{\mathbb{N}}
\newcommand{\one}{\mathbbm{1}}

\newcommand{\Ad}[1]{\textrm{Ad}\left(#1\right)}
\newcommand{\Aut}[1]{\textrm{Aut}\left(#1\right)}

\newcommand{\qe}{\underset{\text{q.e.}}{\sim}}

\newcommand{\Mod}[1]{\mathrm{mod} #1}

\theoremstyle{definition}
\newtheorem{theorem}{Theorem}[section]
\newtheorem{definition}[theorem]{Definition}
\newtheorem{lemma}[theorem]{Lemma}
\newtheorem{remark}[theorem]{Remark}

\numberwithin{equation}{section}
\begin{document}
\section{Setup and definitions}
Take $\omega\in\PP(\AA)$ to be such that
\begin{enumerate}
	\item  there exists an automorphism $\alpha\in\Aut{\AA}$ such that
	\begin{align}
		\omega&=\omega_0\circ\alpha&\alpha&=\Ad{V_1}\circ\alpha_L\otimes\alpha_R\circ\Theta
	\end{align}
	where $\Theta\in\Aut{\AA_{(C_\theta\cup \tau(C_\theta))^c}}$\footnote{By this I mean it is supported in the complement of the union of the horizontally oriented cone through the origin with angle $\theta$ and the cone that is shifted one site upwards (through site $(0,1)$)}, $\alpha_L\in\Aut{\AA_L}$, $\alpha_R\in\Aut{\AA_R},V_1\in\UU(\AA)$ and $\omega_0\in\PP(\AA)$ satisfies the split property.
	\item there exists a map
	\begin{equation}
		\tilde\beta:G\rightarrow\Aut{\AA}:g\mapsto\tilde\beta_g
	\end{equation}
	satisfying
	\begin{align}
		\omega\circ\tilde\beta_g&=\omega&\tilde\beta_g=\Ad{V_2}\circ\eta^L_g\otimes\eta^R_g\circ\beta^U_g
	\end{align}
	where $V_2\in\UU(\AA),\eta^L_g\in\Aut{\AA_L\cap\AA_{C_\theta}}$ and $\eta^R_g\in\Aut{\AA_R\cap\AA_{C_\theta}}$.
\end{enumerate}
Take $(\HH_0=\HH_L\otimes\HH_R,\pi_0=\pi_L\otimes\pi_R,\Omega)$ a GNS triple of $\omega_0$. Now use lemma 2.1 from Yoshiko Ogata \cite{ogata2021h3gmathbb} to define the objects $W_g$ and $u_\sigma(g,h)$ they will be the starting point for our definition. Now take $\tau\in\Aut{\AA}$ to be the automorphism that vertically translates every element of the $C^*$ algebra by one site upwards. We now define the translation action on the GNS space of $\omega_0$:
\begin{lemma}
	There exists a unique $v\in\UU(\HH_0)$ such that
	\begin{enumerate}
		\item $\Ad{v}\circ\pi_0=\pi_0\circ\alpha_0\circ\Theta\circ\tau\circ\Theta^{-1}\circ\alpha_0^{-1}.$
		\item $\pi_0(V_1)v\pi_0(V_1^\dagger)\Omega_0=\Omega_0.$
	\end{enumerate}
\end{lemma}
\section{Definition of the index}
We now have to define one more object
\begin{lemma}
	There exist $K_g^L\in\UU(\HH_L)$ and $K_g^R\in\UU(\HH_R)$ such that
	\begin{equation}
		v^\dagger W_g v W_g^\dagger=K_g^L\otimes K_g^R=K_g
	\end{equation}
	and they are unique (up to a phase). They satisfy the identity
	\begin{equation}
		\Ad{K^R_g}\circ\pi_0=\pi_0\circ \alpha_0\circ \tau^{-1}\circ \eta_g^R\circ\beta_g^{RU}\circ\tau\circ(\beta_g^{RU})^{-1}\circ(\eta_g^{R})^{-1}\circ\alpha_0^{-1}.\quad\footnote{Observe in particular that $\Theta$ has cancelled out}
	\end{equation}
\end{lemma}
\begin{proof}
	content...
\end{proof}
\begin{lemma}\label{lem:AdjointOverConeIsInCone}
	Take $\gamma^{L/R}\in\Aut{(\AA_{C_\theta}\cup\tau(\AA_{C_\theta}))\cap\AA_{L/R}}$ and take $\Gamma^{L/R}$ to be such that
	\begin{equation}
		\pi_{L/R}\circ\alpha_{L/R}\circ\gamma^{L/R}\circ\alpha_{L/R}^{-1}= \Ad{\Gamma^{L/R}}\circ\pi_{L/R}
	\end{equation}
	then
	\begin{equation}
		\Ad{W_g}(\Gamma_{L}\otimes\Gamma_{R})=\Ad{W_g}(\Gamma_L)\otimes\Ad{W_g}(\Gamma_{R}).
	\end{equation}
	Moreover there exist $\tilde\gamma^{L/R}_g\in\Aut{(\AA_{C_\theta}\cup\AA_{C_\theta'})\cap\AA_{L/R}}$ such that
	\begin{equation}
		\pi_{L/R}\circ\alpha_{L/R}\circ\tilde\gamma^{L/R}_g\circ\alpha_{L/R}^{-1}= \Ad{\Ad{W_g}(\Gamma^{L/R})}\circ\pi_{L/R}.
	\end{equation}
\end{lemma}
\begin{proof}
	content...
\end{proof}
\begin{lemma}\label{lem:Definition2Cochain}
	There exists a $C:G^2\rightarrow U(1)$ such that
	\begin{equation}\label{eq:Definition2Cochain}
		K_g^R\Ad{W_g}\left(K_h^R\Ad{W_hW_{gh}^\dagger}\left((K_{gh}^R)^\dagger\right)\right)u_
		R(g,h)=C(g,h)v^\dagger u_R(g,h)v
	\end{equation}
	for all $g,h\in G.$
\end{lemma}
\begin{proof}
	Since the GNS representation is irreducible this is equivalent to showing that the left and righthandside of equation \eqref{eq:Definition2Cochain} have the same adjoint action on the GNS representation. We first prove the result for the full tensor product:
	\begin{align}
		&\Ad{v^\dagger u_L(g,h)v\otimes v^\dagger u_R(g,h) v}\circ\pi_0\\
		&=\Ad{v^\dagger (u_L(g,h)\otimes u_R(g,h)) v}\circ\pi_0\\
		&=\Ad{v^\dagger W_g W_h W_{gh}^{-1}v}\circ\pi_0\\
		&=\Ad{K_gW_g v^\dagger W_h W_{gh}^{-1}v}\circ\pi_0\\
		&=\Ad{K_gW_g K_h W_h v^\dagger W_{gh}^{-1}v}\circ\pi_0\\
		&=\Ad{K_gW_g K_h W_h W_{gh}^\dagger K_{gh}^\dagger }\circ\pi_0\\
		&=\Ad{K_gW_g K_h W_h W_{gh}^\dagger K_{gh}^\dagger W_{gh}W_h^\dagger W_g^\dagger W_gW_hW_{gh}^\dagger}\circ\pi_0\\
		&=\Ad{K_g\Ad{W_g}\left(K_h\Ad{W_hW_{gh}^\dagger}\left(K_{gh}^\dagger\right)\right)u_L(g,h)\otimes u_
			R(g,h)}\circ\pi_0.
	\end{align}
	Using lemma \ref{lem:AdjointOverConeIsInCone} we get that
	\begin{align}
		&(K_g^L\otimes K_g^R)\Ad{W_g}\left((K_h^L\otimes K_h^R)\Ad{W_hW_{gh}^\dagger}\left((K_{gh}^L\otimes K_{gh}^R)^{-1}\right)\right)\\
		&=K_g^L\Ad{W_g}\left(K_h^L\Ad{W_hW_{gh}^\dagger}\left((K_{gh}^L)^\dagger\right)\right)\otimes K_g^R\Ad{W_g}\left(K_h^R\Ad{W_hW_{gh}^\dagger}\left((K_{gh}^R)^\dagger\right)\right)
	\end{align}
	concluding the proof.
\end{proof}
\begin{lemma}
	The function $C$ as defined in lemma \ref{lem:Definition2Cochain} is a 2-cochain.
\end{lemma}
\begin{proof}
	Take lemma 2.4 from \cite{ogata2021h3gmathbb} there it is stated that there exists a 3-cochain $C'$ such that
	\begin{equation}\label{eq:defintion3CochainProof2Cochain}
		u_R(g,h)u_R(gh,k)u_R(g,hk)^\dagger\Ad{W_g}(u_R(h,k)^\dagger)=C'(g,h,k)\mathbbm{1}.
	\end{equation}
	It is clear that this 3-cochain is invariant under the substitution
	\begin{align}
		W_g&\rightarrow v^\dagger W_g v&u_R(g,h)&\rightarrow v^\dagger u_R(g,h)v.
	\end{align}
	If we now prove that it is also invariant under the substitution
	\begin{align}\label{eq:SubstitutionForProofCochain}
		W_g&\rightarrow K_g W_g&u_R(g,h)&\rightarrow K_g^R\Ad{W_g}\left(K_h^R\Ad{W_hW_{gh}^\dagger}\left((K_{gh}^R)^\dagger\right)\right)u_
		R(g,h)
	\end{align}
	we have proved our result because using equation \eqref{eq:Definition2Cochain} we then get that
	\begin{equation}
		C'(g,h,k)=C(h,k)C(gh,k)^{-1}C(g,hk)C(g,h)^{-1}C'(g,h,k)
	\end{equation}
	proving the result. Inserting substitution \eqref{eq:SubstitutionForProofCochain} into \eqref{eq:defintion3CochainProof2Cochain} gives
	\begin{align}
		&u_R(g,h)u_R(gh,k)u_R(g,hk)^\dagger\Ad{W_g}\left(u_R(h,k)^\dagger\right)\\
		\rightarrow&K_g^R\Ad{W_g}\left(K_h^R\Ad{W_hW_{gh}^\dagger}\left((K_{gh}^R)^\dagger\right)\right)u_
		R(g,h)\\
		&K_{gh}^R\Ad{W_{gh}}\left(K_k^R\Ad{W_kW_{ghk}^\dagger}\left((K_{ghk}^R)^\dagger\right)\right)u_
		R(gh,k)\\
		&u_R(g,hk)^\dagger \Ad{W_g}\left( \Ad{W_{hk}W_{ghk}^\dagger}(K^R_{ghk})(K^R_{hk})^{\dagger} \right)(K^R_g)^\dagger\\
		&\Ad{W_g}\left(u_R(h,k)^\dagger \Ad{W_h}\left(\Ad{W_kW_{hk}^\dagger}\left(K_{hk}^R\right)(K^R_k)^\dagger\right)(K^R_h)^\dagger\right).
	\end{align}
	Using the fact that $W_gW_hW_{gh}^\dagger=u_L\otimes u_R(g,h)$ one now gets
	\begin{align}
		=&K_g^R\Ad{W_g}\left(K_h^R\Ad{W_hW_{gh}^\dagger}\left((K_{gh}^R)^\dagger\right)\right)W_gW_hW_{gh}^\dagger u_L(g,h)^\dagger\\
		&K_{gh}^R\Ad{W_{gh}}\left(K_k^R\Ad{W_kW_{ghk}^\dagger}\left((K_{ghk}^R)^\dagger\right)\right) W_{gh}W_kW_{ghk}^\dagger u_L(gh,k)^\dagger\\
		&W_{ghk}W_{hk}^\dagger W_{g}^\dagger u_L(g,hk) \Ad{W_g}\left( \Ad{W_{hk}W_{ghk}^\dagger}(K^R_{ghk})(K^R_{hk})^{\dagger} \right)(K^R_g)^\dagger\\
		&\Ad{W_g}\left(u_R(h,k)^\dagger \Ad{W_h}\left(\Ad{W_kW_{hk}^\dagger}\left(K_{hk}^R\right)(K^R_k)^\dagger\right)(K^R_h)^\dagger\right).
	\end{align}
	We will now use the fact that the $u_L$ commutes with everything that has support only on the right to get
	\begin{align}
		=&K_g^R\Ad{W_g}\left(K_h^R\Ad{W_hW_{gh}^\dagger}\left((K_{gh}^R)^\dagger\right)\right)W_gW_hW_{gh}^\dagger u_L(g,h)^\dagger\\
		&K_{gh}^R\Ad{W_{gh}}\left(K_k^R\Ad{W_kW_{ghk}^\dagger}\left((K_{ghk}^R)^\dagger\right)\right) W_{gh}W_kW_{ghk}^\dagger u_L(gh,k)^\dagger\\
		&\Ad{W_{ghk}W_{hk}^\dagger W_{g}^\dagger}  \left(\Ad{W_g}\left( \Ad{W_{hk}W_{ghk}^\dagger}(K^R_{ghk})(K^R_{hk})^{\dagger} \right)(K^R_g)^\dagger\right)\\
		&u_R(g,hk)^\dagger\Ad{W_g}\left(u_R(h,k)^\dagger \Ad{W_h}\left(\Ad{W_kW_{hk}^\dagger}\left(K_{hk}^R\right)(K^R_k)^\dagger\right)(K^R_h)^\dagger\right)\\
		=&K_g^R\Ad{W_g}\left(K_h^R\Ad{W_hW_{gh}^\dagger}\left((K_{gh}^R)^\dagger\right)\right)W_gW_hW_{gh}^\dagger u_L(g,h)^\dagger\\
		&K_{gh}^R\Ad{W_{gh}}\left(K_k^R\Ad{W_kW_{ghk}^\dagger}\left((K_{ghk}^R)^\dagger\right)\right)  \\
		&\Ad{W_{gh}W_kW_{hk}^\dagger W_{g}^\dagger}  \left(\Ad{W_g}\left( \Ad{W_{hk}W_{ghk}^\dagger}(K^R_{ghk})(K^R_{hk})^{\dagger} \right)(K^R_g)^\dagger\right)\\
		&u_R(gh,k)u_R(g,hk)^\dagger\Ad{W_g}\left(u_R(h,k)^\dagger \Ad{W_h}\left(\Ad{W_kW_{hk}^\dagger}\left(K_{hk}^R\right)(K^R_k)^\dagger\right)(K^R_h)^\dagger\right)\\
		=&K_g^R\Ad{W_g}\left(K_h^R\Ad{W_hW_{gh}^\dagger}\left((K_{gh}^R)^\dagger\right)\right)W_gW_hW_{gh}^\dagger \\
		&K_{gh}^R\Ad{W_{gh}}\left(K_k^R\Ad{W_kW_{ghk}^\dagger}\left((K_{ghk}^R)^\dagger\right)\right)  \\
		&\Ad{W_{gh}W_kW_{hk}^\dagger W_{g}^\dagger}  \left(\Ad{W_g}\left( \Ad{W_{hk}W_{ghk}^\dagger}(K^R_{ghk})(K^R_{hk})^{\dagger} \right)(K^R_g)^\dagger\right)\\
		&W_{gh}W_h^\dagger W_g^\dagger u_R(g,h) u_R(gh,k)u_R(g,hk)^\dagger\Ad{W_g}\left(u_R(h,k)^\dagger\right)\\ &\Ad{W_g}\left(\Ad{W_h}\left(\Ad{W_kW_{hk}^\dagger}\left(K_{hk}^R\right)(K^R_k)^\dagger\right)(K^R_h)^\dagger\right)\\
		=&C'(g,h,k)K_g^R\Ad{W_g}\left(K_h^R\Ad{W_hW_{gh}^\dagger}\left((K_{gh}^R)^\dagger\right)\right)W_gW_hW_{gh}^\dagger \\
		&K_{gh}^R\Ad{W_{gh}}\left(K_k^R\Ad{W_kW_{ghk}^\dagger}\left((K_{ghk}^R)^\dagger\right)\right)  \\
		&\Ad{W_{gh}W_kW_{hk}^\dagger W_{g}^\dagger}  \left(\Ad{W_g}\left( \Ad{W_{hk}W_{ghk}^\dagger}(K^R_{ghk})(K^R_{hk})^{\dagger} \right)(K^R_g)^\dagger\right)\\
		&W_{gh}W_h^\dagger W_g^\dagger\Ad{W_g}\left(\Ad{W_h}\left(\Ad{W_kW_{hk}^\dagger}\left(K_{hk}^R\right)(K^R_k)^\dagger\right)(K^R_h)^\dagger\right).
	\end{align}
	\clearpage
	\begin{align}
		&u_R(g,hk)^\dagger \Ad{W_g}(u_R(h,k)^\dagger)u_R(g,h)u_R(gh,k)\\
		\rightarrow& u_R(g,hk)^\dagger \Ad{W_g}\left(\Ad{W_{hk}W_{ghk}^\dagger}(K^R_{ghk})(K^R_{hk})^\dagger\right)(K^R_g)^\dagger\\ &\Ad{K_gW_g}\left(u_R(h,k)^\dagger\Ad{W_h}\left(\Ad{W_kW_{hk}^\dagger}(K^R_{hk})(K^R_k)^\dagger\right)(K^R_h)^\dagger\right)\\
		&K_g^R\Ad{W_g}\left(K_h^R\Ad{W_hW_{gh}^\dagger}\left((K_{gh}^R)^\dagger\right)\right)u_R(g,h)\\
		&K_{gh}^R\Ad{W_{gh}}\left(K_k^R\Ad{W_kW_{ghk}^\dagger}\left((K_{ghk}^R)^\dagger\right)\right)u_R(gh,k)
	\end{align}
	On the one hand we have
	\begin{align}
		&u_R(gh,k)u_R(g,hk)^\dagger\\
		\rightarrow&K_{gh}^R\Ad{W_{gh}}\left(K_k^R\Ad{W_kW_{ghk}^\dagger}\left((K_{ghk}^R)^\dagger\right)\right)u_R(gh,k)\\
		&u_R(g,hk)^\dagger \Ad{W_g}\left(\Ad{W_{hk}W_{ghk}^\dagger}(K^R_{ghk})(K^R_{hk})^\dagger\right)(K^R_g)^\dagger\\
		=&K_{gh}^RW_{gh} K^R_kW_kW_{ghk}^\dagger(K^R_{ghk})^\dagger W_{ghk}W_k^\dagger W_{gh}^\dagger u_R(gh,k)\\
		&u_R(g,hk)^\dagger W_gW_{hk}W_{ghk}^\dagger K^R_{ghk}W_{ghk}W_{hk}^\dagger(K^R_{hk})^\dagger W_g^\dagger(K^R_g)^\dagger
	\end{align}
	on the other hand we have
	\begin{align}
		&u_R(g,h)^\dagger \Ad{W_g}(u_R(h,k))\\
		\rightarrow&u_R(g,h)^\dagger \Ad{W_g}\left(\Ad{W_hW_{gh}^\dagger}\left(K^R_{gh}\right)(K^R_h)^\dagger\right)\left(K^R_g\right)^\dagger\\
		&\Ad{W_g}\left(K^R_h\Ad{W_h}\left(K^R_k\Ad{W_kW_{hk}^\dagger}\left((K^R_{hk})^\dagger\right)\right)u_R(h,k)\right)\\
		=&u_R(g,h)^\dagger W_gW_hW_{gh}^\dagger K^R_{gh}W_{gh}W_h^\dagger (K^R_h)^\dagger W_g^\dagger (K^R_g)^\dagger\\
		&W_g K^R_h W_h K^R_k W_kW_{hk}^\dagger (K^R_{hk})^\dagger W_{hk}W_k^\dagger W_h^\dagger u_R(h,k)W_g^\dagger\\
		=&u_L(g,h) K^R_{gh}W_{gh}W_h^\dagger (K^R_h)^\dagger W_g^\dagger (K^R_g)^\dagger\\
		&W_g K^R_h W_h K^R_k W_kW_{hk}^\dagger (K^R_{hk})^\dagger u_L(h,k)^\dagger W_g^\dagger\\
		=&u_L(g,h) K^R_{gh}W_{gh}W_h^\dagger (K^R_h)^\dagger W_g^\dagger (K^R_g)^\dagger (W_gW_hW_{gh}^\dagger W_{gh}W_h^\dagger W_g^\dagger)\\
		&W_g (W_hW_kW_{kh}^\dagger W_{kh}W_k^\dagger W_h^\dagger) K^R_h W_h K^R_k W_kW_{hk}^\dagger (K^R_{hk})^\dagger u_L(h,k)^\dagger W_g^\dagger\\
		=& K^R_{gh}W_{gh}W_h^\dagger (K^R_h)^\dagger W_g^\dagger (K^R_g)^\dagger W_gW_hW_{gh}^\dagger u_L(g,h) W_{gh}W_h^\dagger W_g^\dagger\\
		&W_g W_hW_kW_{kh}^\dagger u_L(h,k)^\dagger W_{kh}W_k^\dagger W_h^\dagger K^R_h W_h K^R_k W_kW_{hk}^\dagger (K^R_{hk})^\dagger  W_g^\dagger\\
		=& K^R_{gh}W_{gh}W_h^\dagger (K^R_h)^\dagger W_g^\dagger (K^R_g)^\dagger W_gW_hW_{gh}^\dagger u_R(g,h)^\dagger\\
		&W_g u_R(h,k) W_g^\dagger W_g W_{kh}W_k^\dagger W_h^\dagger K^R_h W_h K^R_k W_kW_{hk}^\dagger (K^R_{hk})^\dagger  W_g^\dagger
	\end{align}
\end{proof}
\section{Things we have to prove}

\bibliography{TSPT}
\bibliographystyle{plain}
\end{document}