\documentclass[preview=false]{standalone}
\begin{document}

\begin{figure}
	\centering
	\def\s{0.4}
	\resizebox{0.45\textwidth}{!}{%
		\begin{tikzpicture}
			\draw[draw=white,line width=0mm] (1+\s,0) coordinate (a) -- (\s,0) coordinate (b) -- (1+\s,1) coordinate (c);
			
			\fill[fill=green!30!white] (-1*\s,0) -- (-6*\s,-5*\s) -- (-6*\s,5*\s);
			\fill[fill=green!30!white] (1*\s,0) -- (6*\s,-5*\s) -- (6*\s,5*\s);
			\fill[fill=red!30!white] (0,1*\s) -- (4*\s,5*\s) -- (-4*\s,5*\s);
			\fill[fill=red!30!white] (0,-1*\s) -- (4*\s,-5*\s) -- (-4*\s,-5*\s);
			
			\draw[draw=black,line width=0.3mm]
			plot[smooth,samples=2,domain=0:4*\s] (\x,\x+1*\s);
			\draw[draw=black,line width=0.3mm]
			plot[smooth,samples=2,domain=-4*\s:0] (\x,-\x+1*\s);
			\draw[draw=black,line width=0.3mm]
			plot[smooth,samples=2,domain=0:4*\s] (\x,-\x-1*\s);
			\draw[draw=black,line width=0.3mm]
			plot[smooth,samples=2,domain=-4*\s:0] (\x,\x-1*\s);
			\draw[draw=black,line width=0.3mm]
			plot[smooth,samples=2,domain=0:5*\s] (\x+1*\s,\x);
			\draw[draw=black,line width=0.3mm]
			plot[smooth,samples=2,domain=0:5*\s] (\x+1*\s,-\x);
			\draw[draw=black,line width=0.3mm]
			plot[smooth,samples=2,domain=-5*\s:0] (\x-1*\s,-\x);
			\draw[draw=black,line width=0.3mm]
			plot[smooth,samples=2,domain=-5*\s:0] (\x-1*\s,\x);
			
			\draw[draw=black,line width=0.3mm] (1*\s,0) -- (3*\s,0);
			\pic [draw, ->, "$\theta$", angle eccentricity=1.5, angle radius=\s*1.5cm,line width=0.3mm] {angle=a--b--c};
			
			\node at (-4*\s,0) {$\eta^L_g$};
			\node at (4.5*\s,0) {$\eta^R_g$};
			\node at (0,3*\s) {$\Theta$};
	\end{tikzpicture}}
	\caption{This figure indicates the support area of the different automorphisms when there are two translation directions. The angle $\theta$ still has to be smaller then or equal to what was indicated here so that the $\Theta$ and the $\eta_g$ commute but now we must be able to do both a horizontal and a vertical widening of the support of $\eta_g$.}
	\label{fig:SetupForTwoTranslations}
\end{figure}
\end{document}